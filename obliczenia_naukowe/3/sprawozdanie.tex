\documentclass[15pt, a4paper]{article}
\usepackage[T1]{fontenc}
\usepackage[polish]{babel}
\usepackage[utf8]{inputenc}
\usepackage{tikz}
\usepackage{adjustbox}
\usepackage{longtable}
\usepackage{graphicx}
\usepackage{amsfonts}
\title{Obliczenia naukowe}
\author{Felix Zieliński 272336}
\date{Lista 2}
\begin{document}
\maketitle

\noindent TESTY STETTY SRESTY
wykresy 
nwm jakies wnioski, czym sie roznia te metody czy cos, nazwy returnowanych w kodzie

\vspace{0.5cm}

\noindent\hrulefill

\vspace{0.5cm}

% zadanie 1

\noindent\textbf{Zadanie 1.} Funkcja rozwiązująca równanie \( f(x) = 0 \) metodą bisekcji.\\\\
\noindent Metoda bisekcji to inaczej metoda równego podziału lub metoda połowienia. 
Korzysta ona z faktu, że funkcja ciągła w przedziale \( [a, b] \), która zmienia w nim swój znak (a więc \( f(a) * f(b) < 0 \)), musi mieć miejsce zerowe w \( (a, b) \).\\\\
\noindent Jeżeli \( f(a) * f(b) < 0 \), to wiem, że gdzieś na przedziale jest miejsce zerowe. Obliczam więc takie \(c\), że \( c = 1/2 * (a + b) \) (połowa przedziału) i sprawdzam z tego samego warunku, czy jest tam miejsce zerowe. Jeżeli tak, to podstawiam \( b = c \), a w przeciwmym razie \( a = c \).\\\\
\noindent Powtarzam to, dopóki nie znajdę zera (bo \( f(a) * f(c) = 0 \) lub \(f(b) * f(c) = 0 \)). \\\\
\noindent W ten sposób jednak, otrzymam tylko jedno miejsce zerowe, nawet jeśli jest ich więcej na tym odcinku.\\\\ 
\noindent Innym warunkiem zakończenia jest warunek \(|f(c)| < \epsilon \) lub \(|b - a| < \delta \). Te stałe są podane przy wywołaniu funkcji i decydują o dokładności wyniku, bowiem dla typu zmennoprzecinkowego \(Float64\) mogą oczywiście nastąpić błędy przybliżeń. \( \epsilon \) jest wartością błędu przybliżenia, a \(\delta\) - pożądaną bliskością otrzymanej wartości iloczynu do zera. 

\vspace{0.5cm}

\noindent\hrulefill

\vspace{0.5cm}

% zadanie 2 

\noindent\textbf{Zadanie 2.} Funkcja rozwiązująca równanie \( f(x) = 0 \) metodą Newtona.\\\\
\noindent To inaczej metoda stycznych. Działa ona dla funkcji, która w przedziale \([a, b]\) musi znajdować się dokładnie jeden jej pierwiastek,  \(f(a) * f(x) < 0 \) oraz jej pierwsza i druga pochodna mają stały znak w przedziale \( [a, b] \).\\\\
\noindent Liczę punkty przecięcia stycznych do funkcji z osią \( OX \), zaczynając od prostej stycznej w \( f(x_0) \). Współrzędnia \(x\), w której styczna przecina oś \(OX\), jest przybliżeniem pierwiastka funkcji. Szukam dalej przybliżeń, aż w końcu któreś spełni dane założenia. 

\vspace{0.5cm}

\noindent\hrulefill

\vspace{0.5cm}

% zadanie 3

\noindent\textbf{Zadanie 3.} Funkcja rozwiązująca równanie \( f(x) = 0 \) metodą siecznych.\\\\
\noindent To inaczej metoda cięciw lub Eulera. Działa ona dla funkcji, która jest dwukrotnie rózniczkowalna na przedziale \([a, b]\) oraz pierwiastek szukany musi być nieparzystej krotności. \\\\
\noindent W tej metodzie używa się ilorazu różnicowego zamiast pochodnej \(f'(x_n) \approx \frac{f(x_n) - f(x_{n-1})}{x_n - x_{n-1}}\). Sama metoda siecznych opisana jest wzorem \(x_{n+1} = x_n - f(x_n) \frac{x_n - x_{n-1}}{f(x_n) - f(x_{n-1})}\) dla dodatnich n.\\\\
\noindent Wyznaczam miejsca przecięć siecznych funkcji z osią \(OX\), rozpoczynając od siecznej mającej swój początek w punkcie \((x_0, f(x_0))\) oraz koniec w \((x_1, f(x_1))\). Sieczna ta przecina oś \(OX\) w \(x_2\), którego używam do wyznaczenia kolejnej siecznej i jej przecięcia z osią. \\\\
\noindent Szukanie przybliżenia pierwiastka kończy się, gdy przybliżenie zera będzie odpowiednio małe (zgodne z podanym do funkcji), bądź gdy różnica między kolejnymi przybliżeniami będzie wystarczająco mała (zgodna z podanym do funkcji).

\vspace{0.5cm}

\noindent\hrulefill

\vspace{0.5cm}

%zadanie 4

\noindent\textbf{Zadanie 4} W tym zadaniu należało wyznaczyć pierwiastki funkcji \(\sin x - \left(\frac{1}{2}x\right)^2\) przy użyciu metod zaprogramowanych w poprzednich zadaniach.\\\\
\noindent Wyniki dla poszczególnych metod przedstawia poniższa tabela:


\begin{longtable}{|c|c|c|c|c|}
    \hline
    \textbf{Metoda} & \textbf{x0} & \textbf{f(x0)} & \textbf{Iteracje} & \textbf{Czy błąd} \\
    \hline
    Medota bisekcji & 1.9337539672851562 & -2.7027680138402843e-7 & 16 & false \\
    \hline
    Metoda Newtona & 1.933753779789742 & -2.2423316314856834e-8 & 4 & 0 \\
    \hline
    Metoda siecznych & 1.933753644474301 & 1.564525129449379e-7 & 4 & false \\
    \hline
\end{longtable}

\noindent Metoda bisekcji potrzebowała najwiecej iteracji, bo aż 16, podczas gdy pozostałe metody potrzebowały ich tylko 4. Metoda Newtora oraz siecznych mają mniejszą złożoność. Każda z metod obliczyła wartość pierwiastka z podobną dokładnością.


\end{document}
