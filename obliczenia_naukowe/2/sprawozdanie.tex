\documentclass[15pt, a4paper]{article}
\usepackage[T1]{fontenc}
\usepackage[polish]{babel}
\usepackage[utf8]{inputenc}
\usepackage{tikz}
\usepackage{adjustbox}
\usepackage{longtable}
\title{Obliczenia naukowe}
\author{Felix Zieliński 272336}
\date{Lista 2}
\begin{document}
\maketitle
TODO OPIS

\vspace{0.5cm}

\noindent\hrulefill

% zadanie 1

\vspace{0.5cm}

\noindent\textbf{Zadanie 1.} Niewielkie zmiany danych oraz ich wpływ na wyniki obliczeń.\\

\noindent W ramach przypomnienia zadania: na poprzedniej liście  należało obliczyć iloczyny skalarne dwóch wektorów na cztery rózne sposoby.\\\\
Zaimplementowałem każdy z podanych w poleceniu sposobów, tak więc funkcja \verb|a| liczy "w przód", od pierwszych indeksów, funkcja \verb|b| "w tył", analogicznie, a \verb|c| oraz \verb|d| liczą, odpowiednio, od największego do najmniejszego oraz od najmniejszego do największego względem ich wartości absolutnej.

\begin{table}[ht]
    \begin{adjustbox}{max width=\textwidth}
    \begin{tabular}{|c|c|c|c|}
        \hline 
        Sposób & Float32 & Float64 & Wartość prawidłowa \\ \hline
        1 & -0.3472038161853561  & 1.0251881368296672e-10 & -1.00657107000000e-11 \\ \hline
        2 & -0.3472038162872195 & -1.5643308870494366e-10 & -1.00657107000000e-11 \\ \hline
        3 & -0.3472038162872195 & 0.0 & -1.00657107000000e-11  \\ \hline
        4 & -0.3472038162872195 & 0.0 & -1.00657107000000e-11 \\ \hline
    \end{tabular}
    \end{adjustbox}
    \label{tab:products}
\end{table}

\vspace{0.5cm}

\end{document}
