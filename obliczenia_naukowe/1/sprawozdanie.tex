% TODO: exponent bias
\documentclass[15pt, a4paper]{article}
\usepackage[T1]{fontenc}
\usepackage[polish]{babel}
\usepackage[utf8]{inputenc}
\usepackage{tikz}
\usepackage{adjustbox}
\usepackage{longtable}
\title{Obliczenia naukowe}
\author{Felix Zieliński 272336}
\date{Lista 1}
\begin{document}
\maketitle
Rozwiązania zadań z 1. listy na przedmiot Obliczenia Naukowe. Programy zostały napisane w języku Julia oraz, gdy było to konieczne, w C.

\vspace{0.5cm}

\noindent\hrulefill

% zadanie 1

\vspace{0.5cm}

\noindent\textbf{Zadanie 1.} 

\textbf{a.} Wyznaczanie iteracyjne epsilonów maszynowych wraz z porównaniem z wartościami zwracanymi przez funkcję esp() oraz z danymi z headera float.h jezyka C.

\vspace{0.5cm}

Iteracyjnie dzielę wartość zmiennej \verb|macheps| przez dwa, zaczynając od wartości 1 w danym typie zmiennoprzecinkowym, dopóki warunek pętli while \verb|1 + macheps > 1| jest spełniony. 

\begin{table}[ht]
    \begin{adjustbox}{max width=\textwidth}
    \begin{tabular}{|c|c|c|c|}
        \hline 
        Typ zmiennoprzecinkowy & Wyznaczona wartość macheps & eps() & <float.h> \\ \hline
        16 & 0.000977 & 0.000977 & brak \\ \hline
        32 & 1.1920929e-7 & 1.1920929e-7 & 1.1920929e-07  \\ \hline
        64 & 2.220446049250313e-16 & 2.220446049250313e-16 &  2.2204460492503131e-16 \\ \hline
    \end{tabular}
    \end{adjustbox}
    \label{tab:macheps}
\end{table}

Moje wyniki zgadzają się z prawdziwymi wartościami, co wskazuje na to, że metoda iteracyjna jest dość dokładna. \\
Wyznaczony epsilon maszynowy pomaga w ustaleniu precyzji zapisu liczb zmiennoprzecinkowych, gdyż jest odległością od 1 do następnej liczby możliwej do zaprezentowania w danym typie. Im mniejsza będzie ta wartość, tym większa będzie precyzja względna obliczeń. 

\vspace{0.5cm}

\textbf{b. } Wyznaczenie iteracyjnie liczby maszynowej eta wraz z porównaniem z wartościami zwracanymi przez funkcję nextfloat()

\vspace{0.5cm}

Iteracyjnie dzielę wartości zmiennej \verb|eta| przez dwa począwszy od wartości zmiennej równej 1, dopóki warunek pętli while \verb|eta > 0| jest spełniony. 

\begin{table}[ht]
    \begin{tabular}{|c|c|c|}
        \hline 
        Typ zmiennoprzecinkowy & Wyznaczona wartośc eta & nextfloat() \\ \hline
        16 & 6.0e-8 & 6.0e-8 \\ \hline
        32 & 1.0e-45 & 1.0e-45 \\ \hline
        64 & 5.0e-324 & 5.0e-324 \\ \hline
    \end{tabular}
    \label{tab:eta}
\end{table}


Wartości zwrócone przez
\begin{enumerate}
    \item \verb|floatmin(Float32)| - 1.1754944e-38
    \item \verb|floatmin(Float64)| - 2.2250738585072014e-308
\end{enumerate}

Tutaj również moje wyniki zgadzają się z prawdziwymi wartościami. \\
Liczba eta odpowiada najmniejszej zdenormalizowanej liczbie dodatniej reprezentowanej w podanej arytmetyce zmiennopozycyjnej. Jest zdenormalizowana, czyli bity cechy mają wartość 0. \\
Natomiast wartości zwrócone przez \verb|floatmin| są odpowiednikiem tej wartości, ale znormalizowanej, dlatego jej wartości są większe.

\vspace{0.5cm}

\textbf{c. } Wyznaczenie iteracyjne liczby MAX wraz z porównaniem z wartościami zwracanymi przez funckje floatmax() oraz z danymi z headera float.h języka C.

\vspace{0.5cm}

Iteracyjnie mnożę wartości zmiennej \verb|max|, aż stanie się ona równa wartości \verb|isinf|. Następnie, w celu poprawienia dokładności obliczeń, dodaję do poprzedniej wartości \verb|max| $\frac{x}{k}$, gdzie \verb|k = 2, 4, ...|, aż \verb|max| będzie równa \verb|isinf|.

\begin{table}[ht]
    \begin{adjustbox}{max width=\textwidth}
    \begin{tabular}{|c|c|c|c|}
        \hline 
        Typ zmiennoprzecinkowy & Wyznaczona wartośc max & floatmax() & <float.h> \\ \hline
        16 & 6.55e4 & 6.55e4 & brak \\ \hline
        32 & 3.4028235e38 & 3.4028235e38 & 3.40282347e+38 \\ \hline
        64 & 1.7976931348623157e308 & 1.7976931348623157e308 & 1.7976931348623157e+308 \\ \hline
    \end{tabular}
    \end{adjustbox}
    \label{tab:max}
\end{table}

Moje wyniki zgadzają się z wartościami zwróconymi przez \verb|floatmax()| oraz z headera \verb|float.h|.

\vspace{0.5cm}

\noindent\hrulefill

% zadanie 2

\vspace{0.5cm}

\noindent\textbf{Zadanie 2.} Sprawdzenie, czy twierdzenie Kahana jest poprawne.

\vspace{0.5cm}

Twierdzenie to mówi, że epsilon maszynowy można uzyskać, obliczając wartość \[3 * (4 / 3 - 1) - 1\] w odpowiedniej arytmetyce zmiennoprzecinkowej. Sprawdzenia dokonuję obliczając tę wartość dla wartości rzutowanych na podany typ. Jedynkę otrzymuję funkcją \verb|one|.

\begin{table}[ht]
    \begin{adjustbox}{max width=\textwidth}
    \begin{tabular}{|c|c|c|}
        \hline 
        Typ zmiennoprzecinkowy & Wyznaczona wartość & eps() \\ \hline
        16 & -0.000977 & 0.000977 \\ \hline
        32 & 1.1920929e-7 & 1.1920929e-7 \\ \hline
        64 & -2.220446049250313e-1 & 2.220446049250313e-16 \\ \hline
    \end{tabular}
    \end{adjustbox}
    \label{tab:kahan}
\end{table}

% TODO: sprawdzenie, czy nie pisze glupot ponizej

Obliczone wyniki praktycznie pokrywają się z prawdziwymi wartościami - nie licząc znaku. Zmiana znaku dla typów \verb|Float16| oraz \verb|Float64| wynika z ilości bitów znaczących w tych typach (odpowiednio, 10 oraz 52). Ponadto, rozwijając 4/3 binarnie, otrzymamy 1.(10). To powoduje, że ostatnią cyfrą mantysy w tych typach będzie 0, co zmienia znak na przeciwny. Tak więc, gdy weźmiemy moduł z obliczonych wartości, otrzymamy poprawne wyniki, więc twierdzenie Kahana w rzeczywistości jest poprawne.

\vspace{0.5cm}

\noindent\hrulefill

% zadanie 3

\vspace{0.5cm}

\noindent\textbf{Zadanie 3.} Sprawdzenie, czy w arytmetyce \verb|Float(64)| liczby zmiennopozycyjne sa równomiernie rozmieszczone.

\vspace{0.5cm}

% TODO: tu tez sprawdzic, czy to nie glupota

Najprostszym, a zarazem najdłuższym obliczeniowo rozwiązaniem jest iteracja przez wszystkie liczby z danego przedziału w celu porównania z wartościami \verb|nextfloat|. Szybciej jednak jest porównać cechy pierwszej oraz ostatniej liczby z przedziału. Jeśli byłyby inne, wykluczyłoby to równomierny rozkład liczb w tym przedziale.

Funkcja potwierdza, że cechy są takie same. Ponadto wypisałem kilka początkowych liczb, co dalej potwierdza równomierne rozmieszczenie:
\[0011111111110000000000000000000000000000000000000000000000000000\]
\[0011111111110000000000000000000000000000000000000000000000000001\]
\[0011111111110000000000000000000000000000000000000000000000000010\]
\[0011111111110000000000000000000000000000000000000000000000000011\]
\[0011111111110000000000000000000000000000000000000000000000000100\]

\vspace{0.5cm}

Wiem, że przesunięcie cechy dla Float64 wynosi 1023. Mantysa ma 52 bity znaczące. Z tych informacji mogę wyliczyć, jak rozmieszczone są liczby w danym przedziale. Używam do tego wzoru $2^{\mathrm{cecha - 1023}}$ * $2^{\mathrm{-52}}$. Wyliczone z tego wzoru kroki prezentują się następująco:
\[[0.5, 1] - 1.1102230246251565e-16\]
\[[1, 2] - 2.220446049250313e-16\]
\[[2, 4] - 4.440892098500626e-16\]

\vspace{0.5cm}

Odległości pomiędzy kolejnymi liczbami rosną wraz ze zwiększaniem się cechy, co jest zgodne ze standardem IEEE754, w którym liczby są reprezentowane z dokładnością różną w zależności od przedziału. Dokładność ta jest tym większa, im bliższe zeru są liczby.

\vspace{0.5cm}

\noindent\hrulefill

% zadanie 4 

\vspace{0.5cm}

\noindent\textbf{Zadanie 4.} Znalezienie w arytmetyce \verb|Float(64)| liczbę zmiennopozycyjną \verb|x| w przedziale \verb|1 < x < 2| taką, że $x * (1/x) \neq 1$, a także najmniejszej takiej liczby.\\
Iteracyjnie sprawdzam kolejne wyniki działania $x * (1/x) \neq 1$, iterując po \verb|x| przy użyciu funkcji \verb|nextfloat|. Gdy tylko wynik będzie różny od 1, zwracam go, tym samym otrzymując najmniejszą wartość w zadanym przedziale.\\
Najmniejsza znaleziona przeze mnie liczba w przedziale (1, 2): \[\verb|1.000000057228997|\]
Niepoprawny wynik działania jest spowodowany niedokładnością, jaką są obarczone działania na liczbach zmiennoprzecinkowych. Trzeba zachować więc ostrożność podczas wykonywania obliczeń i brać pod uwagę, że mogą być one nieidealne, bądź tak je przekształcać, aby obliczenie ich nie stanowiło problemu.

\vspace{0.5cm}

\noindent\hrulefill

% zadanie 5  

\vspace{0.5cm}

\noindent\textbf{Zadanie 5.} Obliczanie iloczynu skalarnego dwóch wektorów na cztery rózne sposoby.\\
Zaimplementowałem każdy z podanych w poleceniu sposobów, tak więc funkcja \verb|a| liczy "w przód", od pierwszych indeksów, funkcja \verb|b| "w tył", analogicznie, a \verb|c| oraz \verb|d| liczą, odpowiednio, od największego do najmniejszego oraz od najmniejszego do największego względem ich wartości absolutnej.

\begin{table}[ht]
    \begin{adjustbox}{max width=\textwidth}
    \begin{tabular}{|c|c|c|c|}
        \hline 
        Sposób & Float32 & Float64 & Wartość prawidłowa \\ \hline
        1 & -0.3472038161853561  & 1.0251881368296672e-10 & -1.00657107000000e-11 \\ \hline
        2 & -0.3472038162872195 & -1.5643308870494366e-10 & -1.00657107000000e-11 \\ \hline
        3 & -0.3472038162872195 & 0.0 & -1.00657107000000e-11  \\ \hline
        4 & -0.3472038162872195 & 0.0 & -1.00657107000000e-11 \\ \hline
    \end{tabular}
    \end{adjustbox}
    \label{tab:products}
\end{table}

\vspace{0.5cm}

Jak widać w tabeli, żaden ze sposobów liczenia nie dał dokładnego wyniku, niezależnie od zastosowanej arytmetyki (\verb|Float32| bądź \verb|Float64|, chociaż ten drugi pozwolił na uzyskanie wyniku bliższego prawdziwej wartości). Widać również, że kolejność wykonywania działań wpływa na wynik.

\vspace{0.5cm}

\noindent\hrulefill

% zadanie 6 

\vspace{0.5cm}

% TODO: wiarygosnosc wyniku
\noindent\textbf{Zadanie 6.} Obliczanie wartości funkcji w arytmetyce \verb|Float64| dla kolejnych wartości argumentu x. Zadane funkcje:
\[
f(x) = \sqrt{x^2 + 1} - 1
\]

\[
g(x) = \frac{x^2}{\sqrt{x^2 + 1} + 1}
\]

Powyższe funkcje zaimplementowałem w niezmienionej formie, używając \verb|Float64|.


\begin{table}[ht]
    \begin{tabular}{|c|c|c|}
        \hline 
        Wartość x & f(x) & g(x) \\ \hline
        $8^{\mathrm{-1}}$ & 0.0077822185373186414 & 0.0077822185373187065 \\ \hline
        $8^{\mathrm{-2}}$ & 0.00012206286282867573 & 0.00012206286282875901 \\ \hline
        $8^{\mathrm{-3}}$ & 1.9073468138230965e-6 & 1.907346813826566e-6 \\ \hline
        $8^{\mathrm{-4}}$ & 2.9802321943606103e-8 & 2.9802321943606116e-8 \\ \hline
        $8^{\mathrm{-5}}$ & 4.656612873077393e-10 & 4.6566128719931904e-10 \\ \hline
        $8^{\mathrm{-6}}$ & 7.275957614183426e-12 & 7.275957614156956e-12 \\ \hline
        $8^{\mathrm{-7}}$ & 1.1368683772161603e-13 & 1.1368683772160957e-13 \\ \hline
        $8^{\mathrm{-8}}$ & 1.7763568394002505e-15 & 1.7763568394002489e-15 \\ \hline
        $8^{\mathrm{-9}}$ & 0.0 & 2.7755575615628914e-17 \\ \hline
        $8^{\mathrm{-10}}$ & 0.0 & 4.336808689942018e-19 \\ \hline
    \end{tabular}
    \label{tab:fvsg}
\end{table}

Jak można zaobserwować, funkcja \verb|f| już przy $8^{-9}$ jest równa 0. Do tego momentu, obie funkcje mają zbliżone wartości. Funkcja \verb|f| traci dokładność przez operowanie na małych liczbach - odejmowanie od pierwiastka, który w pewnym momencie zostaje przybliżony do 1, wartości 1. Możnaby temu zapobiec poprzez odpowiednie przekształcenie równania - np z \verb|f(x)| do \verb|g(x)|, co pozwoliłoby na zachowanie większej prezycji obliczeń.

\vspace{0.5cm}

\noindent\hrulefill

% zadanie 7 

\vspace{0.5cm}

\noindent\textbf{Zadanie 7.} Obliczenie przybliżonej wartości pochodnej w punkcie x, używając wzoru: \\
\[f'(x_0) \approx \tilde{f}'(x_0) = \frac{f(x_0 + h) - f(x_0)}{h}\] \\
Funkcja, której pochodną należało obliczyć, to 
\[f(x) = sinx + cos3x\]
w punkcie \verb|x = 1| oraz błędów przybliżeń. Pochodna jest obliczana ze wzoru:
\[f'(x) = cosx - 3sin3x \] \\
Obliczona dokłada wartość pochodnej w $x_0 = 1$ to
\[0.11694228168853815\]

\vspace{0.5cm}

Zaimplementowałem funkcję, która oblicza wartość funkcji \verb|f(x)| dla zadanej wartości \verb|x|, a także taką liczącą przybliżoną wartość pochodnej oraz błąd. Dokładna wartość pochodnej jest przechowywana w zmiennej globalnej.

\begin{longtable}{|c|c|c|c|}
    \hline
    Wartość \( h \) & Wartość \( 1 + h \) & pochodna & błąd \\ \hline
    \endfirsthead

    \(2^{-0}\) & 2.0 & 2.0179892252685967 & 1.9010469435800585 \\ \hline
    \(2^{-1}\) & 1.5 & 1.8704413979316472 & 1.753499116243109 \\ \hline
    \(2^{-2}\) & 1.25 & 1.1077870952342974 & 0.9908448135457593 \\ \hline
    \(2^{-3}\) & 1.125 & 0.6232412792975817 & 0.5062989976090435 \\ \hline
    \(2^{-4}\) & 1.0625 & 0.3704000662035192 & 0.253457784514981 \\ \hline
    \(2^{-5}\) & 1.03125 & 0.24344307439754687 & 0.1265007927090087 \\ \hline
    \(2^{-6}\) & 1.015625 & 0.18009756330732785 & 0.0631552816187897 \\ \hline
    \(2^{-7}\) & 1.0078125 & 0.1484913953710958 & 0.03154911368255764 \\ \hline
    \(2^{-8}\) & 1.00390625 & 0.1327091142805159 & 0.015766832591977753 \\ \hline
    \(2^{-9}\) & 1.001953125 & 0.1248236929407085 & 0.007881411252170345 \\ \hline
    \(2^{-10}\) & 1.0009765625 & 0.12088247681106168 & 0.0039401951225235265 \\ \hline
    \(2^{-11}\) & 1.00048828125 & 0.11891225046883847 & 0.001969968780300313 \\ \hline
    \(2^{-12}\) & 1.000244140625 & 0.11792723373901026 & 0.0009849520504721099 \\ \hline
    \(2^{-13}\) & 1.0001220703125 & 0.11743474961076572 & 0.0004924679222275685 \\ \hline
    \(2^{-14}\) & 1.00006103515625 & 0.11718851362093119 & 0.0002462319323930373 \\ \hline
    \(2^{-15}\) & 1.000030517578125 & 0.11706539714577957 & 0.00012311545724141837 \\ \hline
    \(2^{-16}\) & 1.0000152587890625 & 0.11700383928837255 & 6.155759983439424e-5 \\ \hline
    \(2^{-17}\) & 1.0000076293945312 & 0.11697306045971345 & 3.077877117529937e-5 \\ \hline
    \(2^{-18}\) & 1.0000038146972656 & 0.11695767106721178 & 1.5389378673624776e-5 \\ \hline
    \(2^{-19}\) & 1.0000019073486328 & 0.11694997636368498 & 7.694675146829866e-6 \\ \hline
    \(2^{-20}\) & 1.0000009536743164 & 0.11694612901192158 & 3.8473233834324105e-6 \\ \hline
    \(2^{-21}\) & 1.0000004768371582 & 0.1169442052487284 & 1.9235601902423127e-6 \\ \hline
    \(2^{-22}\) & 1.000000238418579 & 0.11694324295967817 & 9.612711400208696e-7 \\ \hline
    \(2^{-23}\) & 1.0000001192092896 & 0.11694276239722967 & 4.807086915192826e-7 \\ \hline
    \(2^{-24}\) & 1.0000000596046448 & 0.11694252118468285 & 2.394961446938737e-7 \\ \hline
    \(2^{-25}\) & 1.0000000298023224 & 0.116942398250103 & 1.1656156484463054e-7 \\ \hline
    \(2^{-26}\) & 1.0000000149011612 & 0.11694233864545822 & 5.6956920069239914e-8 \\ \hline
    \(2^{-27}\) & 1.0000000074505806 & 0.11694231629371643 & 3.460517827846843e-8 \\ \hline
    \(2^{-28}\) & 1.0000000037252903 & 0.11694228649139404 & 4.802855890773117e-9 \\ \hline
    \(2^{-29}\) & 1.0000000018626451 & 0.11694222688674927 & 5.480178888461751e-8 \\ \hline
    \(2^{-30}\) & 1.0000000009313226 & 0.11694216728210449 & 1.1440643366000813e-7 \\ \hline
    \(2^{-31}\) & 1.0000000004656613 & 0.11694216728210449 & 1.1440643366000813e-7 \\ \hline
    \(2^{-32}\) & 1.0000000002328306 & 0.11694192886352539 & 3.5282501276157063e-7 \\ \hline
    \(2^{-33}\) & 1.0000000001164153 & 0.11694145202636719 & 8.296621709646956e-7 \\ \hline
    \(2^{-34}\) & 1.0000000000582077 & 0.11694145202636719 & 8.296621709646956e-7 \\ \hline
    \(2^{-35}\) & 1.0000000000291038 & 0.11693954467773438 & 2.7370108037771956e-6 \\ \hline
    \(2^{-36}\) & 1.000000000014552 & 0.116943359375 & 1.0776864618478044e-6 \\ \hline
    \(2^{-37}\) & 1.000000000007276 & 0.1169281005859375 & 1.4181102600652196e-5 \\ \hline
    \(2^{-38}\) & 1.000000000003638 & 0.116943359375 & 1.0776864618478044e-6 \\ \hline
    \(2^{-39}\) & 1.000000000001819 & 0.11688232421875 & 5.9957469788152196e-5 \\ \hline
    \(2^{-40}\) & 1.0000000000009095 & 0.1168212890625 & 0.0001209926260381522 \\ \hline
    \(2^{-41}\) & 1.0000000000004547 & 0.116943359375 & 1.0776864618478044e-6 \\ \hline
    \(2^{-42}\) & 1.0000000000002274 & 0.11669921875 & 0.0002430629385381522 \\ \hline
    \(2^{-43}\) & 1.0000000000001137 & 0.1162109375 & 0.0007313441885381522 \\ \hline
    \(2^{-44}\) & 1.0000000000000568 & 0.1171875 & 0.0002452183114618478 \\ \hline
    \(2^{-45}\) & 1.0000000000000284 & 0.11328125 & 0.003661031688538152 \\ \hline
    \(2^{-46}\) & 1.0000000000000142 & 0.109375 & 0.007567281688538152 \\ \hline
    \(2^{-47}\) & 1.000000000000007 & 0.109375 & 0.007567281688538152 \\ \hline
    \(2^{-48}\) & 1.0000000000000036 & 0.09375 & 0.023192281688538152 \\ \hline
    \(2^{-49}\) & 1.0000000000000018 & 0.125 & 0.008057718311461848 \\ \hline
    \(2^{-50}\) & 1.0000000000000009 & 0.0 & 0.11694228168853815 \\ \hline
    \(2^{-51}\) & 1.0000000000000004 & 0.0 & 0.11694228168853815 \\ \hline
    \(2^{-52}\) & 1.0000000000000002 & -0.5 & 0.6169422816885382 \\ \hline
    \(2^{-53}\) & 1.0 & 0.0 & 0.11694228168853815 \\ \hline
    \(2^{-54}\) & 1.0 & 0.0 & 0.11694228168853815 \\ \hline
\end{longtable}

\vspace{0.5cm}

Jak można zauważyć, najlepsza dokładność jest dla \verb|h = 2^{-28}|, gdyż rząd wielkości błędu jest najmniejszy (wynosi \verb|10^{-9}|). Potem błąd zaczyna rosnąć, aż będzie on wielkości dokładnej wartości pochodnej. Dzieje się tak, ponieważ bardzo małe liczby w formacie zmiennoprzecinkowym mają niewystarczającą liczbę cyfr znaczących, co powoduje utratę dokładności podczas obliczeń, zwłaszcza podczas odejmowania liczb o niewielkiej różnicy w wartości. 

\noindent\textbf{Wnioski ogólne}

Przy obliczeniach na liczbach zmiennopozycyjnych należy zachować szczególną ostrożność. Może dojść do dużych zaokrągleń, a co za tym idzie - błędów w obliczeniach, zwłaszcza przy działaniach na liczbach sobie bliskich. Warto więc się zastanowić nad przekształceniem równania, bądź zmianą typu zmiennej.

\end{document}
