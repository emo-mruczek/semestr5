\documentclass[15pt, a4paper]{article}
\usepackage[T1]{fontenc}
\usepackage[polish]{babel}
\usepackage[utf8]{inputenc}
\usepackage{tikz}
\usepackage{adjustbox}
\usepackage{longtable}
\usepackage{graphicx}
\usepackage{amsfonts}
\usepackage{algpseudocode}
\usepackage{algorithm}
\usepackage{amsmath}
\title{Obliczenia naukowe}
\author{Felix Zieliński 272336}
\date{Lista 4}
\begin{document}
\maketitle

\vspace{0.5cm}

ZMIENIC TROCHE WYGLAD WZOROW W 1.

\noindent\hrulefill

\vspace{0.5cm}

% zadanie 1

\noindent\textbf{Zadanie 1.} W tym zadaniu należało napisać funkcję obliczającą ilorazy różnicowe (bez użycia macierzy).\\\\
Ilorazem różnicowym n-tego rzędu funkcji \(f: X \rightarrow Y \) dla \(x_0, x_1, ..., x_n\) jest funkcja: \( f(x_0, x_1, \dots, x_n) := \sum_{i=0}^{n} \frac{f(x_i)}{\prod_{\substack{j=0 }}^{n} (x_i - x_j)} \).\\\\ Jednakże, aby nie używać macierzy w implementacji, użyłem następującej własności tego iloczynu: \\\\\( f[x_0, x_1, x_2, \ldots, x_n] = \frac{f[x_1, x_2, \ldots, x_n] - f[x_0, x_1, \ldots, x_{n-1}]}{x_n - x_0}\)\\\\
Jak widać, mając poprzedni iloczyn, mogę obliczyć kolejny. 

\vspace{0.5cm}

\noindent\hrulefill

\vspace{0.5cm}

% zadanie 2 

\noindent\textbf{Zadanie 2.} W tym zadaniu należało napisać funkcję obliczającą wartość wielomianu interpolacyjnego stopnia n w postaci Newtona \(N_x(x)\) w punkcie \(x = t\), za pomocą uogólnionego algorytmu Hornera. Algorytm powinien mieć złożoność liniową.\\\\
Pierwszą wartość biorę z końca tablicy, kolejne uzyskuję w myśl wzoru:\\\\
\(w_k(x) = f[k] + w_{k+1}(x)*(x - x_k)\), gdzie \(f[k]\) to wartość ilorazu różnicowego.\\\\
Zwracam ostatnią obliczoną wartość.\\\\
Algorytm iteruje przez całą długość wektora, tak więc złożoność obliczeniowa wynosi \(O(n)\).

\vspace{0.5cm}

\noindent\hrulefill

\vspace{0.5cm}

% zadanie 3 

\noindent\textbf{Zadanie 3.} W tym zadaniu należało napisać funkcję obliczająca (w czasie kwadratowym) współczynniki postaci naturalnej wielomianu interpolacyjnego w postaci Newtona, znająć już jego ilorazy różnicowe oraz węzły.\\\\
Korzystam z faktu, ze wartości ilorazów różnicowych oraz wartości węzłów są mi znane.\\\\
Zadany wielomian mnożymy "od końca" (ostatnich potęg). Używam wzoru rekurencyjnego \(w_k(x) = f[k] - x_k * w_{k+1}(x)\), gdzie \(f[k]\) to wartość ilorazu różnicowego. Następnie muszę odpowiednio zauktualizować wartości poprzednich wspólczynników, co czyni ten algorytm algorytmem o złożoności \(O(n^2)\). 

\vspace{0.5cm}

\noindent\hrulefill

\vspace{0.5cm}

% zadanie 3 

\noindent\textbf{Zadanie 4.} W tym zadaniu należało napisać funkcję interpolującą zadaną funkcję na przedziale \([a, b]\) za pomocą wielomianu interpolacyjnego stopnia n w postaci Newtona, oraz rysującą ją i rzeczony wielomian.\\\\
Początkowo, obliczam odległość między węzłami i wartości interpolacji w nich. Potem obliczam ilorazy różnicowe (przy użycu funkcji z zadania 1.) 

\end{document}
