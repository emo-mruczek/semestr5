\documentclass[15pt, a4paper]{article}
\usepackage[T1]{fontenc}
\usepackage[polish]{babel}
\usepackage[utf8]{inputenc}
\usepackage{tikz}
\title{Algorytmy Optymalizacji Dyskretnej}
\author{Felix Zieliński 272336}
\date{Lista 1}
\begin{document}
\maketitle
Wszystkie programy napisałem w języku C. Początkowo reprezentowałem grafy jako macierz sąsiedztwa ze względu na łatwość implementacji, jednakże szybko przekonałem się, że taka reprezentacja jest nieodpowiednia dla dużych danych i proces był zabijany przez brak pamięci. Dlatego też obecnie moje programy używają listy sąsiedztwa do reprezentacji grafów, która nie tylko wymaga mniej zasobów, ale jest także wydajniejsza.
Programy były wywoływane poprzez:
\textit{cat \{plik.txt\} | \{program\} \{argumenty\}}, gdzie argumentem dla wyświetlenia drzewa przeszukań jest 'T' (tylko Zadanie 1).

\vspace{0.5cm}

\noindent\hrulefill

\vspace{0.5cm}

\noindent\textbf{Zadanie 1.} 

\noindent DFS oraz BFS zaimplementowałem na podstawie ich opisów z książki Thomasa H. Cormena. Oba algorytmy tworzą listę sąsiedztwa według kolejności podanych krawędzi, i tak też przeszukują graf (więc nie kierują się numeracją wierzchołków). Kolejność odwiedziń w DFS jest zapisywana post-order. Poniżej prezentuję wyniki dla grafów z polecenia, przypadki skierowane oraz nieskierowane:

\vspace{0.5cm}

% pierwszy

\begin{center}
    \textbf{1. graf, przypadek skierowany}
\end{center}

\begin{center}
    \textbf{BFS} \hspace{4cm} \textbf{DFS}
\end{center}

\begin{center}
\begin{minipage}{0.45\textwidth}
    \begin{tikzpicture}
        \node[circle, draw] (A) at (0, 2) {1};
        \node[circle, draw] (B) at (2, 3) {2};
        \node[circle, draw] (C) at (2, 1) {3};
        \node[circle, draw] (D) at (4, 3) {4};
        \node[circle, draw] (E) at (4, 2) {5};
        \node[circle, draw] (F) at (4, 1) {6};
        
        \draw[->] (A) -- (B);
        \draw[->] (A) -- (C);
        \draw[->] (B) -- (D);
        \draw[->] (C) -- (E);
        \draw[->] (C) -- (F);
    \end{tikzpicture}
\end{minipage}
\hfill
\begin{minipage}{0.45\textwidth}
    \begin{tikzpicture}
        \node[circle, draw] (A) at (0, 2) {1};
        \node[circle, draw] (B) at (2, 3) {2};
        \node[circle, draw] (C) at (2, 1) {3};
        \node[circle, draw] (D) at (4, 3) {4};
        \node[circle, draw] (E) at (4, 2) {5};
        \node[circle, draw] (F) at (4, 1) {6};
        
        \draw[->] (A) -- (B);
        \draw[->] (A) -- (C);
        \draw[->] (B) -- (D);
        \draw[->] (C) -- (E);
        \draw[->] (C) -- (F);
    \end{tikzpicture}
\end{minipage}
\end{center}

\vspace{0.5cm} 

\begin{center}
    \textbf{Kolejność odwiedzin:}
\end{center}

\vspace{0.3cm} 

\noindent
\textbf{BFS:} 1 3 2 6 5 4 \hfill \textbf{DFS:} 6 3 5 4 2 1

\vspace{1.5cm}

% drugi
\begin{center}
    \textbf{1. graf, przypadek nieskierowany}
\end{center}

\begin{center}
    \textbf{BFS} \hspace{4cm} \textbf{DFS}
\end{center}

\begin{center}
\begin{minipage}{0.45\textwidth}
    \begin{tikzpicture}
        \node[circle, draw] (A) at (0, 2) {1};
        \node[circle, draw] (B) at (2, 3) {2};
        \node[circle, draw] (C) at (2, 1) {3};
        \node[circle, draw] (D) at (4, 3) {4};
        \node[circle, draw] (E) at (4, 2) {5};
        \node[circle, draw] (F) at (4, 1) {6};
        
        \draw[->] (A) -- (B);
        \draw[->] (A) -- (C);
        \draw[->] (B) -- (D);
        \draw[->] (C) -- (E);
        \draw[->] (C) -- (F);
    \end{tikzpicture}
\end{minipage}
\hfill
\begin{minipage}{0.45\textwidth}
    \begin{tikzpicture}
        \node[circle, draw] (A) at (0, 2) {1};
        \node[circle, draw] (B) at (2, 3) {2};
        \node[circle, draw] (C) at (2, 1) {3};
        \node[circle, draw] (D) at (4, 3) {4};
        \node[circle, draw] (E) at (4, 2) {5};
        \node[circle, draw] (F) at (4, 1) {6};
        
        \draw[->] (A) -- (C);
        \draw[->] (C) -- (F);
        \draw[->] (D) -- (B);
        \draw[->] (E) -- (D);
        \draw[->] (F) -- (E);
    \end{tikzpicture}
\end{minipage}
\end{center}

\vspace{0.5cm} 

\begin{center}
    \textbf{Kolejność odwiedzin:}
\end{center}

\vspace{0.3cm} 

\noindent
\textbf{BFS:} 1 3 2 6 5 4 \hfill \textbf{DFS:} 2 4 5 6 3 1

\vspace{1.5cm}

% trzeci

\begin{center}
    \textbf{2. graf, przypadek skierowany}
\end{center}

\begin{center}
    \textbf{BFS} \hspace{4cm} \textbf{DFS}
\end{center}

\begin{center}
\begin{minipage}{0.45\textwidth}
    \begin{tikzpicture}
        \node[circle, draw] (A) at (0, 2) {1};
        \node[circle, draw] (B) at (2, 3) {2};
        \node[circle, draw] (C) at (2, 1) {4};
        \node[circle, draw] (D) at (4, 3) {3};
        \node[circle, draw] (E) at (4, 2) {6};
        \node[circle, draw] (F) at (4, 1) {8};
        \node[circle, draw] (G) at (6, 2) {5};
        \node[circle, draw] (H) at (6, 1) {7};

        \draw[->] (A) -- (B);
        \draw[->] (A) -- (C);
        \draw[->] (B) -- (D);
        \draw[->] (B) -- (E);
        \draw[->] (C) -- (F);
        \draw[->] (E) -- (G);
        \draw[->] (F) -- (H);
    \end{tikzpicture}
\end{minipage}
\hfill
\begin{minipage}{0.45\textwidth}
    \begin{tikzpicture}
        \node[circle, draw] (A) at (0, 2) {1};
        \node[circle, draw] (B) at (2, 3) {2};
        \node[circle, draw] (C) at (2, 1) {4};
        \node[circle, draw] (D) at (4, 3) {3};
        \node[circle, draw] (E) at (4, 2) {5};
        \node[circle, draw] (F) at (4, 1) {6};
        \node[circle, draw] (G) at (6, 1) {7};
        \node[circle, draw] (H) at (6, 2) {8};

        \draw[->] (A) -- (B);
        \draw[->] (A) -- (C);
        \draw[->] (C) -- (F);
        \draw[->] (E) -- (D);
        \draw[->] (F) -- (G);
        \draw[->] (H) -- (D);
        \draw[->] (H) -- (G);
    \end{tikzpicture}
\end{minipage}
\end{center}

\vspace{0.5cm} 

\begin{center}
    \textbf{Kolejność odwiedzin:}
\end{center}

\vspace{0.3cm} 

\noindent
\textbf{BFS:} 1 4 2 8 6 3 7 5 \hfill \textbf{DFS:} 3 7 8 4 5 6 2 1

\vspace{1.5cm}

% czwarty

\begin{center}
    \textbf{2. graf, przypadek nieskierowany}
\end{center}

\begin{center}
    \textbf{BFS} \hspace{4cm} \textbf{DFS}
\end{center}

\begin{center}
\begin{minipage}{0.45\textwidth}
    \begin{tikzpicture}
        \node[circle, draw] (A) at (0, 2) {1};
        \node[circle, draw] (B) at (2, 1) {2};
        \node[circle, draw] (C) at (2, 3) {4};
        \node[circle, draw] (D) at (4, 3) {3};
        \node[circle, draw] (E) at (4, 2) {8};
        \node[circle, draw] (F) at (4, 1) {5};
        \node[circle, draw] (G) at (6, 2) {7};
        \node[circle, draw] (H) at (6, 1) {6};

        \draw[->] (A) -- (B);
        \draw[->] (A) -- (C);
        \draw[->] (A) -- (F);
        \draw[->] (F) -- (E);
        \draw[->] (F) -- (H);
        \draw[->] (H) -- (G);
        \draw[->] (C) -- (D);
    \end{tikzpicture}
\end{minipage}
\hfill
\begin{minipage}{0.45\textwidth}
    \begin{tikzpicture}
        \node[circle, draw] (A) at (0, 2) {1};
        \node[circle, draw] (B) at (2, 3) {5};
        \node[circle, draw] (C) at (2, 1) {4};
        \node[circle, draw] (D) at (4, 3) {6};
        \node[circle, draw] (E) at (4, 2) {8};
        \node[circle, draw] (F) at (6, 3) {7};
        \node[circle, draw] (G) at (6, 2) {2};
        \node[circle, draw] (H) at (6, 1) {3};

        \draw[->] (A) -- (B);
        \draw[->] (B) -- (D);
        \draw[->] (D) -- (F);
        \draw[->] (F) -- (E);
        \draw[->] (E) -- (C);
        \draw[->] (C) -- (H);
        \draw[->] (H) -- (G);
    \end{tikzpicture}
\end{minipage}
\end{center}

\vspace{0.5cm} 

\begin{center}
    \textbf{Kolejność odwiedzin:}
\end{center}

\vspace{0.3cm} 

\noindent
\textbf{BFS:} 1 5 4 2 6 8 3 7 \hfill \textbf{DFS:} 2 3 4 8 7 6 5 1

\vspace{1.5cm}

% piaty

\begin{center}
    \textbf{3. graf, przypadek skierowany}
\end{center}

\begin{center}
    \textbf{BFS} \hspace{4cm} \textbf{DFS}
\end{center}

\begin{center}
\begin{minipage}{0.45\textwidth}
    \begin{tikzpicture}
        \node[circle, draw] (A) at (0, 2) {1};
        \node[circle, draw] (B) at (2, 2) {2};
        \node[circle, draw] (C) at (2, 1) {3};
        \node[circle, draw] (D) at (4, 3) {5};
        \node[circle, draw] (E) at (4, 2) {4};
        \node[circle, draw] (F) at (6, 3) {6};
        \node[circle, draw] (G) at (5, 2) {9};
        \node[circle, draw] (H) at (6, 1) {7};
        \node[circle, draw] (I) at (4, 1) {8};

        \draw[->] (A) -- (B);
        \draw[->] (A) -- (C);
        \draw[->] (A) -- (D);
        \draw[->] (B) -- (E);
        \draw[->] (D) -- (F);
        \draw[->] (F) -- (G);
        \draw[->] (F) -- (H);
        \draw[->] (E) -- (I);
    \end{tikzpicture}
\end{minipage}
\hfill
\begin{minipage}{0.45\textwidth}
    \begin{tikzpicture}
        \node[circle, draw] (A) at (0, 2) {1};
        \node[circle, draw] (B) at (2, 3) {2};
        \node[circle, draw] (C) at (2, 1) {3};
        \node[circle, draw] (D) at (4, 3) {5};
        \node[circle, draw] (E) at (4, 2) {4};
        \node[circle, draw] (F) at (6, 3) {6};
        \node[circle, draw] (G) at (8, 2) {9};
        \node[circle, draw] (H) at (6, 1) {7};
        \node[circle, draw] (I) at (8, 1) {8};

        \draw[->] (A) -- (B);
        \draw[->] (A) -- (C);
        \draw[->] (A) -- (D);
        \draw[->] (D) -- (F);
        \draw[->] (F) -- (G);
        \draw[->] (F) -- (H);
        \draw[->] (H) -- (E);
        \draw[->] (H) -- (I);
    \end{tikzpicture}
\end{minipage}
\end{center}

\vspace{0.5cm} 

\begin{center}
    \textbf{Kolejność odwiedzin:}
\end{center}

\vspace{0.3cm} 

\noindent
\textbf{BFS:} 1 5 3 2 6 4 7 9 8 \hfill \textbf{DFS:} 8 4 7 9 6 5 3 2 1

\vspace{1.5cm}

% szósty

\begin{center}
    \textbf{3. graf, przypadek nieskierowany}
\end{center}

\begin{center}
    \textbf{BFS} \hspace{4cm} \textbf{DFS}
\end{center}

\begin{center}
\begin{minipage}{0.45\textwidth}
    \begin{tikzpicture}
        \node[circle, draw] (A) at (0, 2) {1};
        \node[circle, draw] (B) at (2, 3) {2};
        \node[circle, draw] (C) at (2, 1) {3};
        \node[circle, draw] (D) at (4, 3) {5};
        \node[circle, draw] (E) at (4, 2) {4};
        \node[circle, draw] (F) at (6, 3) {6};
        \node[circle, draw] (G) at (6, 2) {9};
        \node[circle, draw] (H) at (6, 1) {7};
        \node[circle, draw] (I) at (4, 1) {8};

        \draw[->] (A) -- (B);
        \draw[->] (A) -- (C);
        \draw[->] (A) -- (D);
        \draw[->] (D) -- (F);
        \draw[->] (D) -- (E);
        \draw[->] (D) -- (H);
        \draw[->] (H) -- (I);
        \draw[->] (H) -- (G);
    \end{tikzpicture}
\end{minipage}
\hfill
\begin{minipage}{0.45\textwidth}
    \begin{tikzpicture}
        \node[circle, draw] (A) at (0, 2) {1};
        \node[circle, draw] (B) at (2, 3) {5};
        \node[circle, draw] (C) at (2, 1) {4};
        \node[circle, draw] (D) at (4, 3) {7};
        \node[circle, draw] (E) at (4, 2) {3};
        \node[circle, draw] (F) at (6, 3) {6};
        \node[circle, draw] (G) at (6, 2) {9};
        \node[circle, draw] (H) at (6, 1) {8};
        \node[circle, draw] (I) at (2, 2) {2};

        \draw[->] (A) -- (B);
        \draw[->] (B) -- (D);
        \draw[->] (D) -- (F);
        \draw[->] (F) -- (E);
        \draw[->] (F) -- (G);
        \draw[->] (G) -- (H);
        \draw[->] (H) -- (C);
        \draw[->] (C) -- (I);
    \end{tikzpicture}
\end{minipage}
\end{center}

\vspace{0.5cm} 

\begin{center}
    \textbf{Kolejność odwiedzin:}
\end{center}

\vspace{0.3cm} 

\noindent
\textbf{BFS:} 1 5 3 2 7 4 6 9 8 \hfill \textbf{DFS:} 2 4 8 9 3 6 7 5 1

\vspace{1.5cm}

% siódmy

\begin{center}
    \textbf{Mój graf:}
\end{center}











\end{document}
