\documentclass[15pt, a4paper]{article}
\usepackage[T1]{fontenc}
\usepackage[polish]{babel}
\usepackage[utf8]{inputenc}
\usepackage{tikz}
\usepackage{adjustbox}
\usepackage{longtable}
\usepackage{graphicx}
\usepackage{amsfonts}
\title{Algorytmy Optymalizacji Dyskretnej}
\author{Felix Zieliński 272336}
\date{Lista 2}
\begin{document}
\maketitle

\noindent\hrulefill

\vspace{0.5cm}

% zadanie 1

\noindent\textbf{Zadanie 1.} W tym zadaniu należało zminimalizowad koszty zakupu paliwa poprzez wyznaczenie planu zakupu i dostaw paliwa na lotniska.\\

\noindent\textbf{Uogólnione parametry z zadania:}

\begin{itemize}
    \item \( L_j \) - j-te lotnisko
    \item \( F_i \) - i-ta firma
    \item \( z_j \) - zapotrzebowanie j-tego lotniska
    \item \( p_i \) - podaż paliwa z i-tej firmy
    \item \( k_{ij} \) - koszt zakupu galonu paliwa od i-tej firmy przez j-te lotnisko
\end{itemize}

\noindent\textbf{Zmienne decezyjne:}\\

\(x_{ij}\) - ilość paliwa dostarczona przez i-tą firmę na j-te lotnisko.\\

\noindent\textbf{Ograniczenia:}

\begin{itemize}
    \item \(x_{ij} \geq 0 \) - ilość paliwa musi być nieujemna
    \item \(\sum_{i}x_{ij} = z_j\)  - suma dostaw do danego lotniska musi zaspokoić jego zapotrzebowanie 
    \item \(\sum_{j}x_{ij} \leq p_i \) - firma nie może dostarczyć więcej paliwa, niż sama produkuje
\end{itemize}

\noindent\textbf{Funkcja celu:}\\

Koszt wszystkich dostaw: \(min \sum_{i, j}x_{ij} * k_{ij} \)\\

\noindent\textbf{Rozwiazanie:}\\ 

TBD

\vspace{0.5cm}

\noindent\hrulefill

\vspace{0.5cm}

% zadanie 2

\noindent\textbf{Zadanie 2.} W tym zadaniu należało zmaksymalizować zysk zakładu poprzez wyznaczenie optymalnego tygodniowego planu placy.\\

\noindent\textbf{Uogólnione parametry z zadania:}

\begin{itemize}
    \item \( L_i \) - i-ty wyrób
    \item \( M_j \) - j-ta maszyna
    \item \( cp_{ij} \) - czas (w minutach na kilogram) obróbki i-tego wyroby na j-tej maszynie
    \item \( C_j \) - czas dostępności j-tej maszyny w godzinach
    \item \( sp_i \) - cena sprzedaży i-tego wyrobu
    \item \( kp_j \) - koszt za godzinę pracy j-tej maszyny
    \item \( km_{i} \) - koszt materiałowy za kilogram i-tego wyrobu
    \item \( z_{i} \) - maksymalny tygodniowy popyt na i-ty wyrób
\end{itemize}

\noindent\textbf{Zmienne decezyjne:}\\

\(x_{i}\) - liczba kilogramów wyprodukowanego i-tego wyrobu.\\

\noindent\textbf{Ograniczenia:}

% TODO: MINUTY KUWA
% TODO: koszty produkcji?
% uxglednic, ze to tygodniowe

\begin{itemize}
    \item \(x_{ij} \geq 0 \) - ilość wyprodukowanego wyrobu musi być nieujemna
    \item \(\sum_{i}x_{i} * cp_{ij} \leq C_j / 60\)  - maszyny mają ograniczony czas pracy 
    \item \(x_{i} \leq z_i \) - nie ma sensu produkować więcej wyrobu, niż jest na niego popyt 
\end{itemize}

\noindent\textbf{Funkcja celu:}\\

Zysk, jako różnica między przychodem a kosztami zmiennymi:\\ \(max( x_{i} * ( \sum_{i}( sp_i - km_{i} ) - \sum_{j}(kp_j / 60) * \sum_{i}(cp_{ij} / 60)  ))\)\\

\noindent\textbf{Rozwiazanie:}\\ 

TBD

\vspace{0.5cm}

\noindent\hrulefill

\vspace{0.5cm}

% zadanie 3

\noindent\textbf{Zadanie 3.} W tym zadaniu należało zminimalizować łączny koszt produkcji w firmie poprzez wyznaczenie optymalnego planu produkcji oraz magazynowania.\\

\noindent\textbf{Uogólnione parametry z zadania:}

\begin{itemize}
    \item \( m_j \) - maksymalna produkcja towaru w j-tym okresie (w jednostkach)
    \item \( k_j \) - j-ty okres (w którym wytwarzane jest maksymalnie 100 jednostek towaru)
    \item \( c_j \) - koszt produkcji jednej jednostki towaru w j-tym okresie
    \item \( a_j \) - maksymalna wielkość (w jednostkach) opcjonalnej produkcji ponadwymiarowej w j-tym okresie
    \item \( o_j \) - koszt jednostkowy w j-tej opcjonalnej produkcji ponadwymiarowej
    \item \( d_j \) - zapotrzebowanie na towar w j-tym okresie
    \item \( s \) - maksymalna ilość jednostek możliwa do przechowania z jednego okresu na kolejny
    \item \( sm_j \) - stan magazynu na początku okresu
    \item \( km \) - koszt magazynowania za jednostkę 
    \item \( mp \) - początkowa ilość jednostek w magazynie
\end{itemize}

\noindent\textbf{Zmienne decezyjne:}

\begin{itemize}
    \item \( x_j \) - ilość jednostek wyprodukowanych w j-tym okresie
    \item \( y_j \) - ilość jednostek wyprodukowanych w j-tym okresie w produkcji opcjonalnej
    \item \( z_j \) - ilość jednostek do przechowania na koniec j-tego okresu
\end{itemize}

\noindent\textbf{Ograniczenia:}

\begin{itemize}
    \item \(x_{j} \geq 0 \) - ilość jednostek wyprodukowanych w j-tym okresie musi być nieujemna
    \item \(y_{j} \geq 0 \) - ilość jednostek wyprodukowanych w j-tym okresie w produkcji opcjonalnej musi być nieujemna
    \item \(z_{j} \geq 0 \) - ilość jednostek do przechowania na koniec j-tego okresu musi być nieujemna
    \item \( x_{j} \leq m_j \) - nie można wyprodukować jednostek ponad maksymalną produkcję towaru w j-tym okresie     
    \item \( y_{j} \leq a_j\) - nie można wyprodukować jednostek dodatkowych ponad maksymalną opcjonalną produkcję towaru w j-tym okresie     
    \item \( z_j \leq s \) - nie można przechowywać jednostek ponad maksymalną ilość jednostek możliwą do przechowania z jednego okresu na kolejny
    \item koszt produkcji jednostek opcjonalnych przewyższa koszt produkcji podstawowej, a więc nie ma potrzeby ograniczania wykorzystania wszystkich jednostek przed rozpoczęciem produkcji opcjonalnej
    \item \( s_1 = mp \) - na początku pierwszego okresu stan magazynu jest równy stanowi początkowemu 
    \item \( s_K+1 = 0 \) - na koniec nie powinno zostać jednostek w magazynie
\end{itemize}

\noindent\textbf{Funkcja celu:}\\

Koszt produkcji oraz magazynowania: \(min \sum_{j=1}^{K}(x_{j} * c_{j} + y_{j} * o_{j} + z_{j} * km)\)\\

\noindent\textbf{Rozwiazanie:}\\ 

TBD

\vspace{0.5cm}

\noindent\hrulefill

\vspace{0.5cm}

% zadanie 4

\noindent\textbf{Zadanie 1.} W tym zadaniu należało zminimalizowad koszty zakupu paliwa poprzez wyznaczenie planu zakupu i dostaw paliwa na lotniska.\\

\noindent\textbf{Uogólnione parametry z zadania:}

\begin{itemize}
    \item \( L_j \) - j-te lotnisko
    \item \( F_i \) - i-ta firma
    \item \( z_j \) - zapotrzebowanie j-tego lotniska
    \item \( p_i \) - podaż paliwa z i-tej firmy
    \item \( k_{ij} \) - koszt zakupu galonu paliwa od i-tej firmy przez j-te lotnisko
\end{itemize}

\noindent\textbf{Zmienne decezyjne:}\\

\(x_{ij}\) - ilość paliwa dostarczona przez i-tą firmę na j-te lotnisko.\\

\noindent\textbf{Ograniczenia:}

\begin{itemize}
    \item \(x_{ij} \geq 0 \) - ilość paliwa musi być nieujemna
    \item \(\sum_{i}x_{ij} = z_j\)  - suma dostaw do danego lotniska musi zaspokoić jego zapotrzebowanie 
    \item \(\sum_{j}x_{ij} \leq p_i \) - firma nie może dostarczyć więcej paliwa, niż sama produkuje
\end{itemize}

\noindent\textbf{Funkcja celu:}\\

Koszt wszystkich dostaw: \(min \sum_{i, j}x_{ij} * k_{ij} \)\\

\noindent\textbf{Rozwiazanie:}\\ 

TBD

\vspace{0.5cm}

\noindent\hrulefill

\vspace{0.5cm}







\end{document}
