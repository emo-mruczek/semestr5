\documentclass[15pt, a4paper]{article}
\usepackage[T1]{fontenc}
\usepackage[polish]{babel}
\usepackage[utf8]{inputenc}
\usepackage{tikz}
\usepackage{adjustbox}
\usepackage{longtable}
\usepackage{graphicx}
\usepackage{amsfonts}
\title{Algorytmy Optymalizacji Dyskretnej}
\author{Felix Zieliński 272336}
\date{Lista 2}
\begin{document}
\maketitle


\noindent\hrulefill

\vspace{0.5cm}

% zadanie 1

\noindent\textbf{Zadanie 1.} W tym zadaniu należało zminimalizowad koszty zakupu paliwa poprzez wyznaczenie planu zakupu i dostaw paliwa na lotniska.\\

\noindent\textbf{Uogólnione parametry z zadania:}

\begin{itemize}
    \item \( L_j \) - j-te lotnisko
    \item \( F_i \) - i-ta firma
    \item \( z_j \) - zapotrzebowanie j-tego lotniska
    \item \( p_i \) - podaż paliwa z i-tej firmy
    \item \( k_{ij} \) - koszt zakupu galonu paliwa od i-tej firmy przez j-te lotnisko
\end{itemize}

\noindent\textbf{Zmienne decezyjne:}\\

\(x_{ij}\) - ilość paliwa dostarczona przez i-tą firmę na j-te lotnisko.\\

\noindent\textbf{Ograniczenia:}

\begin{itemize}
    \item \(x_{ij}\) - ilość paliwa musi być nieujemna
    \item \(\sum_{i}x_{ij} = z_j\)  - suma dostaw do danego lotniska musi zaspokoić jego zapotrzebowanie 
    \item \(\sum_{j}x_{ij} \leq p_i \) - firma nie może dostarczyć więcej paliwa, niż sama produkuje
\end{itemize}

\noindent\textbf{Funkcja celu:}\\

Koszt wszystkich dostaw: \(min \sum_{i, j}x_{ij} * k_{ij} \)\\

\noindent\textbf{Rozwiazanie:}\\ 

TBD

\vspace{0.5cm}

\noindent\hrulefill

\vspace{0.5cm}

% zadanie 2

\noindent\textbf{Zadanie 2.} W tym zadaniu należało zmaksymalizować zysk zakładu poprzez wyznaczenie optymalnego tygodniowego planu placy.\\

\noindent\textbf{Uogólnione parametry z zadania:}

\begin{itemize}
    \item \( L_j \) - j-te lotnisko
    \item \( F_i \) - i-ta firma
    \item \( z_j \) - zapotrzebowanie j-tego lotniska
    \item \( p_i \) - podaż paliwa z i-tej firmy
    \item \( k_{ij} \) - koszt zakupu galonu paliwa od i-tej firmy przez j-te lotnisko
\end{itemize}

\noindent\textbf{Zmienne decezyjne:}\\

\(x_{ij}\) - ilość paliwa dostarczona przez i-tą firmę na j-te lotnisko.\\

\noindent\textbf{Ograniczenia:}

\begin{itemize}
    \item \(x_{ij}\) - ilość paliwa musi być nieujemna
    \item \(\sum_{i}x_{ij} = z_j\)  - suma dostaw do danego lotniska musi zaspokoić jego zapotrzebowanie 
    \item \(\sum_{j}x_{ij} \leq p_i \) - firma nie może dostarczyć więcej paliwa, niż sama produkuje
\end{itemize}

\noindent\textbf{Funkcja celu:}\\

Koszt wszystkich dostaw: \(min \sum_{i, j}x_{ij} * k_{ij} \)\\

\noindent\textbf{Rozwiazanie:}\\ 

TBD






\end{document}
