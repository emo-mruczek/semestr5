\documentclass[15pt, a4paper]{article}
\usepackage[T1]{fontenc}
\usepackage[polish]{babel}
\usepackage[utf8]{inputenc}
\usepackage{tikz}
\usepackage{adjustbox}
\usepackage{longtable}
\title{Języki formalne i techniki translacji}
\author{Felix Zieliński 272336}
\date{Zadanie 5 lista 2}
\begin{document}
\maketitle

\vspace{0.5cm}

\noindent\hrulefill

% polecenie

\vspace{0.5cm}

\noindent\textbf{Zadanie 5.} Czy język \( \{ ww^R x : w, x \in \{0,1\}^* \land w, x \neq \varepsilon \} \), gdzie \( w^R \) oznacza odwrócenie kolejności liter w słowie \( w \), jest regularny?

\vspace{0.5cm}

\noindent\textbf{Rozwiązanie}

\vspace{0.5cm}

\noindent\textbf{Szkic}

Co to język regularny
Co to język nieregularny

Wykorzystam lemat o pompowaniu, gdyż używa się go do dowodzenia, że dany język \verb|L| nie jest językiem regularnym. 

lemat definicja

proof by contradiction

czemu zwykly lemat nie dziala,a uogolniony juz tak

opis uogolnionego



\end{document}
